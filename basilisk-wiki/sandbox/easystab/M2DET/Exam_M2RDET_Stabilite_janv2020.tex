% !TEX encoding = UTF-8 Unicode
\documentclass[a4paper,11pt]{article}
\usepackage[francais]{babel}		% typographie franaise (french marche aussi)
%\usepackage[mac]{inputenc}		% lettres accentuées
\usepackage{color}
\usepackage[utf8]{inputenc}
\usepackage{graphicx, amsmath, amssymb}

\setlength{\textheight}{26cm}
\setlength{\topmargin}{-20mm}
\setlength{\textwidth}{18cm}
\setlength{\oddsidemargin}{-10mm} % marge gauche (im)paire = 1in. + x1mm
%\setlength{\evensidemargin}{0mm} % book : marge gauche paire = 1in. + x2mm
\setlength{\parskip}{1.2ex plus 0.9ex minus 0.4ex}

\newcommand{\dd} {\textrm d}   
\newcommand{\im} {\mathrm i}   
\newcommand{\e}  {\textrm e}   
\newcommand{\dpa}[2]  {\frac {\partial #1} {\partial #2}}   
\newcommand{\ddpa}[2] {\frac {\partial^{2} #1} {\partial #2 ^{2}}}   
\newcommand{\dto}[2]  {\frac {{\textrm d} #1} {{\textrm d} #2}}   
\newcommand{\ddto}[2] {\frac {\textrm{d}^{2} #1} {\textrm{d} #2 ^{2}}}

%-----------------------------------------------------------------------
\begin{document}
{\Large
\noindent
%M2R DET,

\begin{center}
{\bf Examen du cours d'instabilités, M2 DET \\
 30 janvier 2020} \\
{\it \small Durée 2 heures. La qualité de la présentation de la copie sera prise en compte. 
%Barme indicatif : I : 3 points, II 5 points, III : 12 points. %Présentation : 2 points.
} \\
{\it \small Documents autorisés : tous documents manuscrits ou imprimés, livres non autorisés.}
\end{center}
}

%\medskip
%-----------------------------------------------------------------------
%\textbf{I. Questions de cours}
%


\section{Instabilité d'un écoulement cisaillé à surface libre}

On considère un écoulement parallèle d'un liquide homogène de masse volumique $\rho$, défini par $\vec u = \bar{U}(y) \vec{e}_x
$ et associé à un champ de pression $\bar{P}(y)$, occupant l'espace $y \in [0,H]$. La position $y=0$ est un mur, et la position $y= \eta(x) = H$ une surface libre. La région $y>H$ est occupée par un gaz de masse volumique et viscosité négligeables, à la pression uniforme $P_a$.

La gravité est (sauf dans la partie D) supposée dirigée selon la verticale descendante : $\vec g = - g \vec e_y$ avec $g >0$.

Le liquide est supposé non non visqueux (sauf dans la partie E). 
 



 \begin{enumerate}
 
 \item { \bf Question préliminaire :} Montrez que le champ de pression $P(y)$ est un champ hydrostatique et donnez son expression.

{\bf A. Equations des perturbations (théorie non visqueuse)}

On étudie des petites perturbations à cet écoulement sous la forme modale suivante.
%\begin{equation}
$$
\left[ \begin{array}{c} u \\ v \\ p \\ \eta \end{array} \right] 
= 
\left[ \begin{array}{c} \bar{U}(y) \\ 0 \\ \bar{P}(y) \\ H \end{array} \right] 
+ 
\left[ \begin{array}{c} \hat{u}(y) \\ \hat{v}(y) \\ \hat{p}(y) \\ \hat{\eta} \end{array} \right] e^{i (k x -\omega t)} 
$$
%\end{equation}

\item Si $k$ est réel, expliquez la signification physique des parties réelles et imaginaires de $\omega$ .

\item 
A partir des équations d'Euler écrire 3 équations linéaires reliant $ \hat{u}(y)$, $\hat{v}(y)$ et $\hat{p}(y)$.

  \item Après avoir introduit une fonction de courant $\hat{\psi}$ (et justifié son existence), monter que ces équations peuvent se combiner pour aboutir à l'équation de Rayleigh (avec $c = \omega/k$):
  
 \begin{equation}
 (\bar{U} - c) (\partial_y^2 - k^2) \hat \psi + \bar{U}'' \hat \psi = 0
 \end{equation}
  
  \item 
  Ecrire deux conditions limites reliant respectivement $\hat{v}(H)$ à $\hat{\eta}$ et $\hat{p}(H)$ à $\hat{\eta}$
   
   \item Montrez que ces deux équations peuvent se combiner pour aboutir à l'expression suivante:
   
 \begin{equation}
  \left( U(H) - c\right)^2  {\left[ \frac{\partial \hat{\psi}}{\partial y} \right] }_{y=H}  - U'(H) \left( U(H) - c\right) \hat{\psi}(H) - g \hat{\psi}(H) = 0.
\end{equation}

\item Précisez également la condition limite vérifiée par $\hat{\psi}$ en $y=0$.


{\bf B. Cas particulier sans écoulement}

\item On suppose que $U(y) = 0$. Résolvez le problème et montrer qu'on retrouve la relation de dispersion classique des ondes de gravité en profondeur finie :
$\omega ^2 = g k \tanh (kH)$.

Dans ce cas particulier le problème est-il stable ou instable ?

 



{\bf C. Théorème de Rayleigh}

\item A partir de l'équation de Rayleigh, en supposant qu'il existe un mode instable vérifiant $c_i > 0$, démontrez l'identité suivante :

$$
 c_i \left\{ \int_{0}^{H} \frac{\bar{U}''(y)}{|\bar{U}(y)-c|^2} |\hat{\psi}(y)|^2 d y 
-  \frac{U'(H) }{|\bar{U}(H)-c|^2} |\hat{\psi}(H)|^2 + \frac{2 g (U(H) - c_r)  }{|\bar{U}(H)-c|^4} |\hat{\psi}(H)|^2
\right\} = 0.
$$

\item 

On suppose que $U'(y) >0$ et $U''(y) <0$ pour tout $y \in [0,H]$.
conclure que si $g=0$ il ne peux exister de mode instable. Ce résultat est-il en accord avec le théorème de Rayleigh ?


%\item Un théorème du à Howard (admis) permet de montrer que s'il existe un mode instable ($c_i >0$) alors la partie réelle de $c$ associée vérifie forcément $c_r  \in [\bar{U}_{min}, \bar{U}_{max}]$  ou $\bar{U}_{min}$ et 
%$\bar{U}_{max}$ sont les valeurs extrémales prises par $\bar{U}(y)$ sur l'intervalle $[0,H]$. En supposant toujours $U'(y) >0$ et $U''(y) <0$ pour tout $y \in [0,H]$, 
%que peut-on en conclure quand à l'existence possible de modes instables ?

%Cette conclusion est-elle en accord avec le théorème de Rayleigh ?


{\bf D. Solution dans le cas d'un cisaillement linéaire et d'une stratification inversée}

On suppose que le profil de vitesse est linéaire :
$\bar{U}(y) = V y/H$ 
Et que la gravité est maintenant orientée dans la direction $+y$ :

$\vec g = + g \vec e_y$ avec $g >0$. 

(En retournant les axes il s'agit donc d'un film liquide cisaillé "suspendu" sous une paroi horizontale).


\item Justifiez que l'équation de Rayleigh (1) est toujours valable et que la condition limite (2) doit être remplacée par :

 $$\left( V - c\right)^2  {\left[ \frac{\partial \hat{\psi}}{\partial y} \right] }_{y=H}  - \frac{V\left( V - c\right)}{H} \hat{\psi}(H) + g \hat{\psi}(H) = 0.
 $$

\item Montrez que les solutions du problème sont données par la relation de dispersion suivante  (où $\omega^{(rel)} = \omega - k V$ est la fréquence relative dans le repère se déplaçant à la vitesse $V$ de la surface libre): 
$$
\left[\omega^{(rel)}\right]^2 + \left[ \frac{V}{H} \omega^{(rel)} + g k \right] \tanh ( kH ) =0
$$

\item  Calculez le discriminant de cette equation et montrez que le problème est instable dans la limite des petites longueurs d'onde ($k\gg 1$).

Quel(s) effets non pris en compte dans l'analyse peuvent-ils modifier cette conclusion ?


\item 
Montrez que si $V>2 \sqrt{gH}$ les grandes longueurs d'ondes sont restabilisées par le cisaillement.
(Indice : on pourra raisonner graphiquement en comparant le comportement des fonctions $4kg$ et $(V/H)^2 \tanh(kH)$.)


{\bf E. Problème visqueux : résolution numérique}

\item A partir des équations de Navier-Stokes, écrire 3 équations linéaires 
reliant $ \hat{u}(y)$, $\hat{v}(y)$ et $\hat{p}(y)$.

\item Ecrire deux conditions-limites à vérifier en $y=0$ et trois conditions-limites à vérifier en $y=H$.
(Indice : la condition dynamique à la surface libre est à remplacer par la continuité des contraintes normales et tangentielles et fournit donc 2 équations). 
  
\item Montrez que le problème peut se mettre sous la forme matricielle $A \hat{q} = - i \omega B \hat{q}$ où $\hat{q} = [ \hat{u}, \hat{v}, \hat{p} , \hat{\eta} ]$. En déduire la structure d'un programme pour résoudre ce problème. 
  
\end{enumerate}



 
 


%\section{Le Brusselator}


%\subsection{Brusselator homogène}

%Le "Brusselator" est un système dynamique modélisant certaines réactions chimiques donnant lieu à des comportements complexes.
%Ce modèle correspond au système suivant :

%$$
%\frac{d x_1}{dt}  = −(\beta +1)x_1+x_1^2 x_2+\alpha,
%$$
%$$
%\frac{d x_2}{dt} = \beta x_1 - x_1^2 x_2;
%$$

%\clearpage

\section{Bifurcations d'une équation modèle}

On considère l'équation modèle suivante :
$$
\frac{ d x}{dt} = r x + A x^3 - x^5
$$

{\bf On considère dans les trois premières questions le le cas $A = 0$.}

\begin{enumerate}

\item Calculez les points d'équilibre de cette équation. Préciser le nombre de solutions en fonction du signe de $r$.

\item Etudiez (par la méthode de votre choix) la stabilité des points d'équilibre identifiés.

\item Tracez un diagramme des bifurcations. A quel type classique de bifurcation la situation étudiée ici est-elle apparentée ?

{\bf On considère pour les trois questions suivantes le cas $A = 1/2$.}

\item Calculez les points d'équilibre, et préciser leur nombre selon les cas $r<-1$, $-1<r<0$ et $r>0$.

\item  Etudiez (par la méthode de votre choix) la stabilité des points d'équilibre identifiés.

\item Tracez un diagramme des bifurcations. Préciser la nature des trois points de bifurcation observés sur ce diagramme.

\end{enumerate}






\end{document}



\item Donner la solution du système différentiel $\dot{\bf x} = A {\bf x}$, ${\bf x}(0) = {\bf x}_0$, pour les deux matrices suivantes : 
\begin{equation}
A = \left(
\begin{array}{cc}
a & 0 \\
0 & b
\end{array} \right) ; \qquad
A = \left(
\begin{array}{cc}
\sigma & -\omega \\
\omega & \sigma
\end{array} \right).
\end{equation}

%\item Soit le système différentiel $\dot{\bf x} = A {\bf x}$, $A = {\rm diag}(\lambda_1, \lambda_2)$. Tracer le portrait de phase pour $(\lambda_1, \lambda_2) = (Ð1,1)$ ; préciser le type du point fixe, et à quelle valeur propre correspond chaque direction propre.

%\item  Soit le système différentiel $\dot{\bf x} = A {\bf x}$, $A = {\rm diag}(\lambda_1, \lambda_2)$. Tracer le portrait de phase pour $(\lambda_1, \lambda_2) = (2,1)$ ; préciser le type du point fixe, et à quelle valeur propre correspond chaque direction propre.

%\item Les systèmes des deux questions précédentes sont-ils conservatifs ?

\item Donner la définition dÕun point fixe hyperbolique. Donner la définition de lÕéquivalence topologique de deux systèmes. Donner la condition de lÕéquivalence topologique dÕun système non linéaire et du système linéarisé en un point fixe.

\item On considère le système différentiel
\begin{eqnarray*}
\dot{x} = x (2 Ð x Ð y), \\
\dot{y} = y (3 Ð x Ð 2y).
\end{eqnarray*}
\begin{enumerate}
\item Déterminer les régions du plan $(x,y)$ o le flot défini par ce système est contractant ou dilatant. 
\item Déterminer les points fixes, leur stabilité, et leur type. 
\item Déterminer les directions propres pour chacun des points fixes. 
\item \'Ebaucher le portrait de phase au voisinage des points fixes. 
\item Compléter le portrait de phase, après avoir tracé le champ de vecteurs $(\dot{x}, \dot{y})$ sur les axes $x = 0$ et $y = 0$, sur les courbes telles que $\dot{x} = 0$ ou $\dot{y} = 0$, et déterminé les régions du plan o $\dot{x} > 0$ et $\dot{y} > 0$, $\dot{x} > 0$ et $\dot{y} < 0$, etc.
\end{enumerate}

\item On considère lÕoscillateur de DŸffing dissipatif 
\begin{equation}
\ddto{x}{t} + \epsilon \dto{x}{t} +  x - a x^3 = 0
\label{eq:0731bis}
\end{equation}
avec $\epsilon$ réel, $a >0$. Déterminer les points fixes dans lÕespace des phases. Déterminer le type du point fixe $(0, 0)$ selon la valeur de $\epsilon$.

%\item Quelle est la condition de stabilité linéaire dÕun point fixe $x_*$ d'un système différentiel $\dot{\bf x} = A {\bf x}$, ${\bf x} \in  \mathbb{R}^n$. Pour $n=2$, représenter les valeurs propres dans le plan complexe (a) pour $x_*$ stable, (b) pour $x_*$ instable.

\item Quelle est la condition de stabilité linéaire dÕun point fixe $x_*$ dÕune application ${\bf x}_{k+1} = f({\bf x}_k)$, ${\bf x} \in  \mathbb{R}^n$ ? Pour $n = 2$, représenter le spectre des valeurs propres dans le plan complexe (a) pour $x_*$ stable, (b) pour $x_*$ instable.

\item Déterminer les points fixes de lÕapplication 
\begin{equation}
x_{k+1} = x_k (1 + \mu - x_k),
\end{equation}
ainsi que leur stabilité, selon la valeur de $\mu$ réel.

\item \`A quelles condition un point fixe dÕune application est-il dit hyperbolique ? Pour quelles valeurs de $\mu$ les points fixes de lÕapplication ci-dessus sont-ils non hyperboliques ?

\item Déterminer les points périodiques de période minimale 2 de lÕapplication ci-dessus.


\end{enumerate}


\end{document} 

%%%%%%%%%%%%%%%%%%%%%%%%%%%%%%%%%%%%%%%%%%%%%%%%%%%%%%%%%%%%%%%%%%%%%%%%

NOM - Prénom :	4 mars 1998

Phénomènes non linéaires Ð Contr™le continu n¡ 2


6. Soit A = BBC[(AARHS2CO 2(a; b;c; d)) la matrice jacobienne dÕune application, calculée en un point fixe. Quelles inégalités les invariants de cette matrice doivent-ils vérifier pour que le point fixe soit un noeud stable ?

%%%%%%%%%%%%%%%%%%%%%%%%%%%%%%%%%%%%%%%%%%%%%%%%%%%%%%%%%%%%%%%%%%%%%%%%

NOM - Prénom :	Date : 18 mars 1998

Physique non linéaire Ð Contr™le continu n¡ 3

1. système O(x;¥) = f(x) : donner la forme normale et le diagramme de bifurcation dÕune bifurcation noeud-col.

2. système O(x;¥) = f(x) : donner la forme normale et le diagramme de bifurcation dÕune bifurcation fourche.

3. Soit le système O(x;¥) = mx Ð x2. Tracer le diagramme de bifurcation et préciser la nature de la bifurcation.

4. système O(w;¥) = f(w ) : Donner la forme normale et le diagramme de bifurcation dÕune bifurcation de Hopf.

5.Quel type de bifurcation lÕapplication xn+1 = mxn Ð xO(n;2)  subit-elle en m = 1.Tracer le diagramme de bifurcation. 

6. On considère une application xn+1 = f(xn) dont un point fixe perd son hyperbolicité pour m = 0 avec l = Ð1. A quel type de bifurcation peut-on sÕattendre ? Quelle condition f(x) doit-elle remplir pour quÕil en soit ainsi ?



%-----------------------------------------------------------------------
\section{Modèle de Lorenz de la convection de Rayleigh-Bénard} \label{sec:lorenz}

Le système de Lorenz est une modélisation extrême de la convection thermique de Rayleigh-Bénard, qui reproduit certains comportements observés expérimentalement, notamment l'apparition de rouleaux de convection au-delà d'une valeur critique du nombre de Rayleigh. Ce système s'écrit
\begin{eqnarray*}
\dot{x} &=& - P x + P y \\
\dot{y} &=& - y + r x - z x  \\
\dot{z} &=& - b z + x y,
\end{eqnarray*}

%où $x$ est une vitesse caractéristique, et $y$ et $z$ deux températures caractéristiques. $P$ est un nombre sans dimension analogue au nombre de Prandtl, $r$ correspond à un nombre de Rayleigh réduit, et $b$ correspond au nombre d'onde de la perturbation. On considère ici $r$ comme le paramètre variable, et les deux autre paramètres fixés aux valeurs classiques $P = 10$ et $b = 8/3$. 

\begin{enumerate}

\item Rappelez le lien entre ce système et la convection de Rayleigh-Bénard. Que représentent les 3 variables dynamiques $x$, $y$, $z$ ? A quoi correspondent les
paramètres $r$, $P$ et $b$ ?

Dans la suite on prendra les valeurs classiques $P = 10$ et $b = 8/3$ et on considèra $r$ comme paramètre de contrôle.

\item Etudiez la stabilité de la solution triviale $[x,y,z]$ = $[0,0,0]$. Montrez que celle-ci subit une bifurcation pour $r>1$. De quel type de bifurcation s'agit-il ?

\item Déterminer les points fixes non triviaux (notés $[x_p,y_p,z_p]$) du système apparaissant pour $r>1$. 

A quelles structures d'écoulement du problème de convection ces solutions correspondent-elles ?

\item Etudiez la stabilité linéaire des points fixes apparaissant pour $r>1$ en posant $[x,y,z] = [x_p,y_p,z_p] + \epsilon [\hat{x},\hat{y},\hat{z}] e^{\lambda t}$ 
avec $\epsilon \ll 1$.

Ecrire le polynôme caractéristique dont les racines sont les valeurs propres $\lambda$.

\item On admettra que le polynôme écrit précédemment a trois solutions, dont l'une est réelle et négative, et les deux autres complexes conjuguées de partie réelle négative lorsque $r < 24,74$ et positive lorsque $r>24.74$. 
A quel type de bifurcation peut-on s'attendre ?
   
\item Décrire en quelques mots la nature des solutions rencontrées pour $r>24.74$ et la signification physique de ces solutions pour le problème de convection.   

\end{enumerate}




\section{Instabilité d'une couche de mélange d'épaisseur nulle entre deux fluides non miscibles de même densité}


On étudie une couche de mélange (discontinuité de vitesse) entre deux fluides {\em non miscibles} 
mais {\em de masse volumique $\rho$ identique} .

Par exemple le demi-espace $y<0$ est rempli d'huile de parafine de vitesse $u =-U$ et le demi-espace $y>0$ est rempli d'alcool de vitesse
$ u = +U$ (deux liquides de densité sensiblement égale).

On note $\gamma$ la tension de surface ; les masses volumiques étant identiques {\em on pourra négliger la gravité }.

On souhaite étudier la stabilité linéaire de perturbations de longueur d'onde $\lambda = 2 \pi /k $ dans la direction $x$.
On suppose pour cela que l'interface est déplacée d'une amplitude $y = \eta(x,t) = C e^{i k x - i \omega t}+ c.c.$.

On admettra que la courbure d'une interface ainsi définie est donnée par $K \approx \partial^2 \eta/ \partial x^2$.
  



\begin{enumerate}

\item Montrez que des perturbations de nombre d'onde $k$ dans la direction $x$ sont gouvernées par la relation de dispersion suivante :
\begin{equation}
(k U + \omega)^2 + ( kU - \omega)^2 - \frac{\gamma}{\rho}  k^3 = 0
\end{equation} 

Vous utiliserez la démarche de votre choix pour établir cette relation mais veillerez à bien préciser et justifier les hypothèses faites dans la modélisation.

\item Représentez graphiquement $\omega_i$ en fonction de $k$ et $c_r = \omega_r/k$ en fonction de $k$.

\item Montrez qu'on a deux régimes différents correspondant à $k<k_c$ et $k>k_c$ avec $k_c = 2 \rho U^2 / \gamma$. 
Interprétez physiquement chacun de ces deux régimes.

\item Calculez la longueur d'onde $\lambda_{\max}$ correspondant au mode le plus amplifié, ainsi que le taux d'amplification $\omega_{i,max}$ correspondant.

\end{enumerate}


\section{Question de cours}

$$
\includegraphics[width=.55\linewidth]{FigureEcoulementsParalleles.png}
$$
\begin{enumerate}

\item
On considère 4 exemples d'écoulement parallèle d'un fluide incompressible représentés par les profils $(a)$, $(b)$, $(c)$, $(d)$ ci-dessus.
Pour chacun des cas, expliquez la signification physique de l'écoulement (dans quels contextes ou applications ce type d'écoulement est-il rencontré)
puis discutez ses propriétés de stabilité. A quel type d'instabilités peut-on s'attendre ?  Argumentez en vous basant sur les notions vues en cours (notamment les critères classiques de stabilité).


\end{enumerate}


\section{Evolution d'une population animale : étude de stabilité}

On étudie l'équation aux dérivées partielles suivante, gouvernant l'évolution de la fonction scalaire $\phi(x,t)$, définie dans l'intervalle $x\in [0,\infty]$ et $t\in [0,\infty]$ :

\begin{equation}
\frac{\partial \phi}{\partial t} + U \frac{\partial \phi}{\partial x} =  \sigma_0 \phi + \mu  \frac{\partial^2 \phi}{\partial x^2}  
\quad \quad 
\end{equation}


%associée aux conditions limites $\phi(0,t) = \phi(\infty,t) = 0$ et à la condition initiale $\phi(x,0) = \phi_0(x)$.

Ce modèle peut décrire par exemple l'évolution d'une colonie de micro-organismes dans une rivière en présence d'une source de nourriture $\sigma_0$ homogène, les bactéries subissant de plus une diffusion due à leurs déplacements aléatoires (modélisée par le coefficient de diffusion $\mu$) ainsi qu'une advection par le courant moyen (de vitesse uniforme $U$).

\begin{enumerate}

\item En recherchant des solutions modales en $x$ et $t$, (proportionnelles à $e^{i kx - i \omega t}$), écrire une relation de dispersion reliant $\omega$ et $k$.

\item En adoptant un point de vue {\em temporel}, ($k$ réel), exprimez $\omega$ en fonction de $k$. Quelles conditions sur les paramètres et sur $k$ conduisent à une instabilité ?

\item En étudiant le {\em point-selle} de cette relation de dispersion, précisez quelles conditions sur les paramètres conduisent à une instabilité convective ou absolue, respectivement.

({\em on admettra que ce point-selle vérifie la propriété de "chemin de descente la plus raide"}).

\item En considérant une population de micro-organismes initialement localisés au voisinage de $x=0$, comment se comporte la colonie dans les cas {\em absolu} et {\em convectif} ?


%\item Dans le cas on l'instabilité est convective, donnez les solutions $k^+(\omega)$ et k^-(\omega)$ du problème de stabilité spatiale ($\omega$ réel).


\end{enumerate}


